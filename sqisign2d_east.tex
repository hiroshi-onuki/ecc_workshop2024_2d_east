% documentclass of Beamer
\documentclass[12pt]{beamer}

% use theme
\usetheme{Madrid}
\setbeamertemplate{theorems}[numbered]
\setbeamertemplate{frametitle}[default][center]

% hyperref setting
\usepackage{hyperref}
\hypersetup{
    colorlinks=true,
    linkcolor=[rgb]{0,0.6,0}
}

% use packages
\usepackage{amsmath,amssymb,amsthm}
\usepackage{mathtools}
\usepackage{color}
\usepackage[all,color]{xy}
\usepackage{graphicx}
\usepackage{ascmac}
\usepackage[linesnumbered,ruled,vlined]{algorithm2e}
\usepackage{tikzsymbols}
\usepackage{totcount}
\usepackage{array}

% define commands
\newcommand{\Z}{\mathbb{Z}}
\newcommand{\Q}{\mathbb{Q}}
\newcommand{\F}{\mathbb{F}}
\newcommand{\Fp}{\mathbb{F}_p}
\newcommand{\Fpp}{\mathbb{F}_{p^2}}
\newcommand{\B}{\mathcal{B}_{p,\infty}}
\newcommand{\qi}{\mathrm{i}}
\newcommand{\qj}{\mathrm{j}}
\newcommand{\polylogp}{\mathrm{polylog}(p)}
\newcommand{\End}{\mathrm{End}}
\newcommand{\order}{\mathcal{O}}
\newcommand{\gen}[1]{\langle #1 \rangle}
\newcommand{\norm}{\mathrm{n}}
\newcommand{\red}[1]{\textcolor{red}{#1}}
\newcommand{\blue}[1]{\textcolor{blue}{#1}}
\newcommand{\gray}[1]{\textcolor{gray}{#1}}

\title[SQIsign2D-East]{SQIsign2D-East}
\author[]{
    Kohei Nakagawa\inst{1}, \underline{Hiroshi Onuki}\inst{2}\\
    Wouter Castryck\inst{3},
    Mingjie Chen\inst{3},
    Riccardo Invernizzi\inst{3},
    Gioella Lorenzon\inst{3},
    and Frederik Vercauteren\inst{3}
    }
\institute[]{
    \inst{1}NTT Social Informatics Laboratories,
    \inst{2}The University of Tokyo,
    \inst{3}COSIC, ESAT, KU Leuven
}
\date{2024.10.**}

\begin{document}

\begin{frame}[plain, noframenumbering]
    \titlepage
\end{frame}

\section{Introduction}

\begin{frame}{Noatation}
    \begin{itemize}
        \setlength{\itemsep}{10pt}
        \item $\lambda$: security parameter.
        \item $p$ : a prime number of the form $2^{a + b} \cdot f - 1 \approx 2^{2\lambda}$,\\[3pt]
                where $b \leq a \leq b + 2$ and $f$ is a small integer.
        \item $E_0 : y^2 = x^3 + x$, a supersingular elliptic curve over $\Fpp$.
        %\item $\B = \Q + \Q \qi + \Q \qj + \Q \qi \qj$ for $\qi^2 = -1$ and $\qj^2 = -p$.
        %\item $\order_0 = \Z + \Z\qi + \Z\frac{\qi + \qj}{2} + \Z\frac{1 + \qi\qj}{2}$: a maximal order of $\B$,\\[3pt]
        %            $\order_0 \cong \End(E_0)$ (we identify $\order_0$ with $\End(E_0)$).
    \end{itemize}

    \vspace{10pt}
    $\End(E_0)$ has "good" properties for isogeny-based cryptography.

    \vspace{20pt}
    All the elliptic curves in this talk are supersingular and over $\Fpp$.
\end{frame}

\section{Algorithmic building blocks}

\newcounter{genRIIcnt}
\regtotcounter{genRIIcnt}
\newcommand*{\genRIItotal}{\total{genRIIcnt}}
\stepcounter{genRIIcnt}
\begin{frame}{generalized RandIsogImages (\arabic{genRIIcnt}/\genRIItotal{})}
    \textbf{What we want to compute:}\\
    \quad An isogeny of a given degree from a \red{general curve $E$} instead of $E_0$.

    \vspace*{10pt}
    \textbf{Idea:}\\
    \quad Use \red{\textsf{SpecialEichlerNorm}} instead of \textsf{FullRepresentInteger}.

    \vspace*{10pt}
    \begin{itembox}{\textsf{SpecialEichlerNorm}}
    \begin{itemize}
        \item Input: an integer $D$ and an elliptic curve $E$.
        \item Output: $\alpha \in \End(E)$ s.t. $\deg\alpha = D$.
    \end{itemize}
    Requirement:
    \begin{itemize}
        \item $\exists$ an isogeny $E_0 \to E$ of prime degree $N$.
        \item $D > pN^3$ and $\left(\frac{-D}{N}\right) = 1$.
    \end{itemize}
    \quad (Typically, $N \approx p^{1/2}$, thus $D > p^{\red{2.5}}$.)
    \end{itembox}
\end{frame}

\stepcounter{genRIIcnt}
\begin{frame}{generalized RandIsogImages (\arabic{genRIIcnt}/\genRIItotal{})}
    \textbf{Problem:}\\
    We need to compute $\alpha \in \End(E)$ s.t. $\deg\alpha = d\cdot(2^{a+b} - d) < p^{\red{2}}$.

    \vspace*{10pt}
    \textbf{Solution:}\\
    Restrict $E$ s.t.
    $\exists$ an isogeny $E_0 \to E$ of prime degree $N < p^{1/3}$.\\[3pt]
    Then we can find $\alpha \in \End(E)$ s.t. $\deg\alpha \approx p\cdot N^3 < p^2$.\\[3pt]
    We also require $\left(\frac{-\deg\alpha}{N}\right) = 1$.
\end{frame}

\stepcounter{genRIIcnt}
\begin{frame}{generalized RandIsogImages (\arabic{genRIIcnt}/\genRIItotal{})}
    \begin{algorithm}[H]
        \caption{\textsf{GenRandIsogImages}}
        \KwIn{
            An elliptic curve $E$
            and an odd integer $d < 2^{a+b}$ s.t.\\
            \hphantom{Input: }
            $\exists$ an isogeny $E_0 \to E$ of prime degree $N$,\\
            \hphantom{Input: }
            \red{$d(2^{a+b} - d) > p\cdot N^3$},
            and \red{$\left(\frac{-d(2^{a+b} - d)}{N}\right) = 1$}.
        }
        \KwOut{A $d$-isogeny $\varphi$ from $E$.}
        \BlankLine
        Let $\alpha \leftarrow \red{\textsf{SpecialEichlerNorm}(E, d\cdot(2^{a+b} - d))}$\;
        Compute a $d$-isogeny $\varphi$ by Kani's lemma\;
        \Return{$\varphi$}\;
    \end{algorithm}
\end{frame}

\newcounter{auxRIIcnt}
\regtotcounter{auxRIIcnt}
\newcommand*{\auxRIItotal}{\total{auxRIIcnt}}
\stepcounter{auxRIIcnt}
\begin{frame}{Auxiliary isogeny (\arabic{auxRIIcnt}/\auxRIItotal{})}
    \vspace{-10pt}
    $$
    \xymatrix{
        E_0 \ar@{..>}[r]^{\varphi_\mathrm{com}} \ar@{..>}[d]_{\varphi_\mathrm{sec}} & E_\mathrm{com} \ar[d]^{\varphi_\mathrm{cha}} \\
        E_\mathrm{pub} \ar[r]_{\varphi_\mathrm{res}} \ar[d]_{\red{\varphi_\mathrm{aux}}} & E_\mathrm{res}\\
        E_\mathrm{aux} & 
    }
    $$
    $\deg\varphi_\mathrm{res} =: q$ and $\deg\varphi_\mathrm{aux} = 2^a - q$.

    \vspace*{10pt}
    \begin{itemize}
        \item $q \approx 2^a \approx p^{1/2} \Rightarrow 2^a - q \approx p^{1/2}$.
        \item The degree of the output of \textsf{GenRandIsogImages} $\approx p$.
        \item We need to compute a \red{shorter} isogeny.
    \end{itemize}
\end{frame}

\stepcounter{auxRIIcnt}
\begin{frame}{Auxiliary isogeny (\arabic{auxRIIcnt}/\auxRIItotal{})}
    $$
        \xymatrix{
            E_\mathrm{pub} \ar[r]^{\blue{\varphi_\mathrm{aux}}} \ar[d] \ar[dr]^{\red{\tau}} & E_\mathrm{aux} \ar[d]^\psi \\
            E_2 \ar[r] & E_1
        }
    $$
    \begin{enumerate}
        \item Compute \red{$\tau$} by \textsf{GenRandIsogImages} with $q(2^a - q) \approx p$.
        \item Divide $\red{\tau} = \psi \circ \blue{\varphi_\mathrm{aux}}$
            s.t. $\deg\psi = q$ and $\deg\blue{\varphi_\mathrm{aux}} = 2^a - q$\\
            by Kani's lemma.
    \end{enumerate}
\end{frame}

\stepcounter{auxRIIcnt}
\begin{frame}{Auxiliary isogeny (\arabic{auxRIIcnt}/\auxRIItotal{})}
    $M(q) \coloneqq q(2^a - q)(2^{a + b} - q(2^a - q))\ (= \deg\alpha \text{ in \textsf{GenRandIsogImages}})$\\

    \vspace{10pt}
    Requirements for $q = \deg\varphi_\mathrm{res}$ and $N = \deg\varphi_\mathrm{sec}$:
    \begin{enumerate}
        \item $q < 2^a$
        \item $q(2^a - q) < 2^{a + b}$\quad (follows from \textbf{1} and $a \leq b + 2$)
        \item $p \cdot N^3 < M(q)$
        \item $\left(\frac{-M(q)}{N}\right) = 1$
    \end{enumerate}
    We say $q$ is \blue{\emph{$(2^a, 2^b, N)$-nice}} if $q$ satisfies the above conditions.

    \vspace{20pt}
    It can be relaxed to
    $q/g$ is $(2^a, 2^b, N)$-nice for some $g \mid f$.\\[3pt]
    \gray{(Recall $p = 2^{a + b} \cdot f - 1$. See our paper for details.)}
\end{frame}


\section{Protocol}

\section{Security}

\section{Conclusion}

\begin{frame}{What's the difference between West and East?}
    $$
    \xymatrix{
        E_0 \ar@{..>}[r]^{\varphi_\mathrm{com}} \ar@{..>}[d]_{\varphi_\mathrm{sec}} & E_\mathrm{com} \ar[d]^{\varphi_\mathrm{cha}} \\
        E_\mathrm{pub} \ar[r]_{\varphi_\mathrm{res}} \ar[d]_{\varphi_\mathrm{aux}} & E_\mathrm{res}\\
        E_\mathrm{aux} & 
    }
    $$

    A. How to compute $\varphi_\mathrm{sec}$, $\varphi_\mathrm{com}$ and $\varphi_\mathrm{aux}$.

    \begin{itemize}
        \item $\varphi_\mathrm{sec}$ and $\varphi_\mathrm{com}$: RandIsogImages (in QFESTA)
        \item $\varphi_\mathrm{aux}$: \red{Generalized RandIsogImages} (new algorithm!)
        \item (challenge and verification are almost the same)
    \end{itemize}
\end{frame}



\begin{frame}{Restriction on $E_\mathrm{pub}$}
    \vspace{-10pt}
    $$
        \xymatrix{
            E_0 \ar[r]^{\varphi_\mathrm{sec}} & E_\mathrm{pub} \ar[r]^{\varphi_\mathrm{aux}} & E_\mathrm{aux}
        }
    $$
    To use \textsf{GenRandIsogImages} for $E_\mathrm{pub}$,
    we require $\deg\varphi_\mathrm{sec} < p^{1/3}$.
    
    \vspace*{10pt}
    Our choice: $\deg\varphi_\mathrm{sec}$ is a random prime $N < p^{1/4 + \epsilon} < p^{1/3}$.
    \begin{itemize}
        \item The number of possible choices of $N$ is about $p^{1/4} \approx 2^{\lambda/2}$.
        \item $N \approx p^{1/4} \approx 2^{\lambda/2} \Rightarrow$ there are about $2^{\lambda/2}$ $N$-isogenies from $E_0$.
        \item The number of possibile choices of $\varphi_\mathrm{sec}$ is about $2^{\lambda}$.
        \item \red{The degree $N$ must be kept secret.}
    \end{itemize}

    \vspace*{10pt}
    Note that
    $\{\text{The output of \textsf{RandIsogImages}}\} \neq \{N\text{-isogenies from }E_0\}$,\\
    so we add a small $\epsilon$ to $1/4$.
\end{frame}

\begin{frame}[plain]{SQIsign2D-East}
    $$
        \xymatrix{
            E_0 \ar@{..>}[r]^{\varphi_\mathrm{com}} \ar@{..>}[d]_{\varphi_\mathrm{sec}} & E_\mathrm{com} \ar[d]^{\varphi_\mathrm{cha}} \\
            E_\mathrm{pub} \ar[r]_{\varphi_\mathrm{res}} \ar[d]_{\varphi_\mathrm{aux}} & E_\mathrm{res}\\
            E_\mathrm{aux} & 
        }
    $$

    \begin{itemize}
        \item $\varphi_\mathrm{sec}$: prime degree $N < p^{1/4 + \epsilon}$ computed by \textsf{RandIsogImages}.
        \item $\varphi_\mathrm{com}$: odd degree $< 2^{a+b}$ computed by \textsf{RandIsogImages}.
        \item $\varphi_\mathrm{cha}$: degree $2^b$ by V\'elu's formula.
        \item $\varphi_\mathrm{res}$: degree $q$ that is $(2^a, 2^b, N)$-nice.
        \item $\varphi_\mathrm{aux}$: degree $2^a - q$ by \textsf{GenRandIsogImages} and Kani's lemma.
    \end{itemize}

    \vspace*{5pt}
    The signature is
    $(E_\mathrm{com}, E_\mathrm{aux},
    \varphi_\mathrm{res}\circ\hat{\varphi}_\mathrm{aux}(P), \varphi_\mathrm{res}\circ\hat{\varphi}_\mathrm{aux}(Q))$\\[3pt]
    for a fixed basix $\{P, Q\}$ of $E_\mathrm{aux}[2^a]$.
\end{frame}

\begin{frame}{CompactSQIsign2D-East}
    $$
        \xymatrix{
            E_0 \ar@{..>}[r]^{\varphi_\mathrm{com}} \ar@{..>}[d]_{\varphi_\mathrm{sec}} & E_\mathrm{com} \ar[d]^{\varphi_\mathrm{cha}} \\
            E_\mathrm{pub} \ar[r]_{\varphi_\mathrm{res}} \ar@{..>}[d]_{\varphi_\mathrm{aux}} & E_\mathrm{res} \ar[d]^{\varphi'_\mathrm{aux}}\\
            E_\mathrm{aux} & E'_\mathrm{aux}
        }
    $$

    \begin{itemize}
        \item $\varphi'_\mathrm{aux}$: degree $2^{f_1} - q$ computed by Kani's lemma.
    \end{itemize}

    \vspace*{10pt}
    The signature is
    $(E'_\mathrm{aux}, \varphi'_\mathrm{aux}\circ\varphi_\mathrm{res}(P), \varphi'_\mathrm{aux}\circ\varphi_\mathrm{res}(Q), \ker\hat{\varphi}_\mathrm{cha})$\\[3pt]
    for a fixed basis $\{P, Q\}$ of $E_\mathrm{pub}[2^a]$.
\end{frame}

\begin{frame}{Signature size}
    \begin{table}
        \begin{center}
            \begin{tabular}{c|c|c}
                Security & Protocol & Signature size (bytes)\\
                \hline
                Level 1 & SQIsign2D-East & 182\\[3pt]
                 & CompactSQIsign2D-East & 150\\[3pt]
                Level 3 & SQIsign2D-East & 271\\[3pt]
                    & CompactSQIsign2D-East & 223\\[3pt]
                Level 5 & SQIsign2D-East & 359\\[3pt]
                    & CompactSQIsign2D-East & 295\\
            \end{tabular}
        \end{center}
    \end{table}

    \vspace*{10pt}
    Smaller than the numbers in the paper\\
    because the compression of the points in the paper is effective.
\end{frame}

\newcounter{attackcnt}
\regtotcounter{attackcnt}
\newcommand*{\attacktotal}{\total{attackcnt}}
\stepcounter{attackcnt}
\begin{frame}{Key recovery attack (\arabic{attackcnt}/\attacktotal{})}
    The attack by Chen-Castryck-Invernizzi-Lorenzon-Vercauteren

    \vspace{10pt}
    \begin{itembox}{Idea}
    \begin{itemize}
        \item The attacker knows $q = \varphi_\mathrm{res}$ from the signature.
        \item Half of prime numbers $N < p^{1/4+\epsilon}$ satisfy $\left(\frac{-M(q)}{N}\right) = 1$.
    \end{itemize}
    $\Rightarrow$ $q$ leaks one bit of information about $\deg\varphi_\mathrm{sec}$.\\[5pt]
    $\Rightarrow$ About $\lambda/2$ signatures are enough to recover $\deg\varphi_\mathrm{sec}$.
    \end{itembox}
\end{frame}

\stepcounter{attackcnt}
\begin{frame}{Key recovery attack (\arabic{attackcnt}/\attacktotal{})}
    The attack procedure:
    \begin{enumerate}
        \item Collect $\lambda/2$ signatures (for the same public key).
        \item For each prime $N < p^{1/4+\epsilon}$, 
            check $\left(\frac{-M(q)}{N}\right) = 1$ for all $q$'s.\\
            (exhaustive search of cost $O(2^{\lambda/2})$)
        \item (Typically) find a unique $N$ satisfying the condition.
        \item Compute all $N$-isogenies from $E_0$. (cost $O(2^{\lambda/2})$)
    \end{enumerate}

    \vspace{10pt}
    The total cost is $O(2^{\lambda/2})$.\\[5pt]
    $\Rightarrow$ The security is reduced to $2^{\lambda/2}$ from $2^\lambda$.
\end{frame}

\newcounter{countermeasurecnt}
\regtotcounter{countermeasurecnt}
\newcommand*{\countermeasuretotal}{\total{countermeasurecnt}}
\stepcounter{countermeasurecnt}
\begin{frame}{Coutermeasure (\arabic{countermeasurecnt}/\countermeasuretotal{})}
    \textbf{Idea:}
    making $q$ independent of $\deg\varphi_\mathrm{sec}$.

    \vspace{10pt}
    \textbf{Solution:}
    \begin{itemize}
        \item $\deg\varphi_\mathrm{sec}$ is sampled from
            a prime $N < p^{1/4+\epsilon}$ s.t. \red{$\left(\frac{3}{N}\right) = -1$}.
        \item Change the condition $\left(\frac{-M(q)}{N}\right) = 1$ to
            \red{$3 \mid M(q)$}.
        \item Modify \textsf{GenRandIsogImages} as shown in the next slide.
            \iffalse
            $\left(\frac{-M(q)}{N}\right) = 1 \Rightarrow$ compute $\alpha \in \End(E_\mathrm{pub})$ s.t. $\deg\alpha = M(q)$.\\[5pt]
            $\left(\frac{-M(q)}{N}\right) = -1$ (so $\left(\frac{-M(q)/3}{N}\right) = 1$)\\
                    $\Rightarrow$ compute $\alpha \in \End(E_\mathrm{pub})$ s.t. $\deg\alpha = M(q)/3$\\
                    \hphantom{$\Rightarrow$} and compute a random $3$-isogeny $\varphi_3: E_\mathrm{pub} \to E_3$.\\
                    \hphantom{$\Rightarrow$} Then, $\deg(\varphi_3 \circ \alpha) = M(q)$
            \fi
    \end{itemize}
\end{frame}

\stepcounter{countermeasurecnt}
\begin{frame}{Coutermeasure (\arabic{countermeasurecnt}/\countermeasuretotal{})}
    \begin{algorithm}[H]
        \caption{(modified) \textsf{GenRandIsogImages}}
        \KwIn{
            An elliptic curve $E$
            and an odd integer $q < 2^a$ s.t.\\
            \hphantom{Input: }
            $\exists$ an isogeny $E_0 \to E$ of prime degree $N$,\\
            \hphantom{Input: }
            $M(q) > p\cdot N^3$,
            \red{$\left(\frac{3}{N}\right) = -1$},
            and \red{$3 \mid M(q)$}.
        }
        \KwOut{A $q(2^a - q)$-isogeny $\varphi$ from $E$.}
        \BlankLine
        \If{$\left(\frac{-M(q)}{N}\right) = 1$}{
            Let $\alpha \leftarrow \textsf{SpecialEichlerNorm}(E, M(q))$\;
        }
        \Else{
            Let $\alpha' \leftarrow \textsf{SpecialEichlerNorm}(E, M(q)\red{/3})$\;
            \red{Compute a random $3$-isogeny $\varphi_3: E \to E_3$}\;
            \red{Let $\alpha \leftarrow \varphi_3 \circ \alpha'$}\;
        }
        Compute a $q(2^a - q)$-isogeny $\varphi$ by Kani's lemma\;
        \Return{$\varphi$}\;
    \end{algorithm}
\end{frame}

\begin{frame}[plain,noframenumbering]
    \begin{center}
        \Huge Thank you for your attention!
    \end{center}
\end{frame}

\end{document}

