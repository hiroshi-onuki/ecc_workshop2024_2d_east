
\begin{frame}{Identification protocol in SQIsign2D-East}

    \vspace{-10pt}
    {\large
    $$
        \xymatrix@C=30pt@R=30pt{
            & E_0 \ar@{..>}[d]_{\varphi_\mathrm{sk}}
                \only<2->{\ar@{..>}[r]^{\varphi_\mathrm{com}}}
                \only<5-6>{\ar@[red]@<7.5pt>[d]}
                & \uncover<2->{E_\mathrm{com}}
                \only<5-6>{\ar@[red]@<7.5pt>[l]}
                \only<3->{\ar[d]^{\varphi_\mathrm{chl}}} \\
            & E_\mathrm{pk}
                \only<5->{\ar[r]_{\varphi_\mathrm{rsp}}}
                \only<5-6>{\ar@[red]@<7.5pt>[r] \ar@{}@<20pt>[r]^{\red{\alpha}}}
                \only<7>{\ar[d]^{\varphi_\mathrm{aux}}}
                \only<7>{\ar[dl]_{\tau}}
                \only<8->{\ar[d]_{\varphi_\mathrm{aux}}}
                & \uncover<3->{E_\mathrm{chl}} \only<5-6>{\ar@[red]@<7.5pt>[u]}\\
            \only<7>{E'} & \uncover<7->{E_\mathrm{aux}} \only<7>{\ar[l]} & 
        }
    $$
    }

    \vspace{10pt}
    \only<1>{
        \textbf{Prover:}
        \begin{itemize}
            \item Take a random odd integer $N_\mathrm{sk} \in [1, p^{1/4}]$ s.t. $\left(\frac{3}{N_\mathrm{sk}}\right) = -1$
                and compute an $N_\mathrm{sk}$-isogeny $\varphi_{\mathrm{sk}}$ by \blue{\textsf{RandIsogImages}}.
            \item The prover will prove the knowledge of $\varphi_{\mathrm{sk}}$ with public $E_\mathrm{pk}$.
        \end{itemize}
    }
    \only<2>{
        \textbf{Prover:}
        \begin{itemize}
            \item Take a random odd integer $N_\mathrm{com} \in [1, 2^\lambda]$\\
                and compute a $N_\mathrm{com}$-isogeny $\varphi_{\mathrm{com}}$ by \blue{\textsf{RandIsogImages}}.
            \item Send $E_{\mathrm{com}}$ to the verifier.
        \end{itemize}
    }
    \only<3>{
        \textbf{Verifier:}
        \begin{itemize}
            \item Compute a $2^b$-isogeny $\varphi_{\mathrm{chl}}$.
            \item Send $\varphi_\mathrm{chl}$ and $E_{\mathrm{chl}}$ to the prover.
        \end{itemize}
    }
    \only<4>{
        \textbf{Prover:}
        \begin{itemize}
            \item Compute the ideal $I$ corresponding to $\varphi_\mathrm{chl}\circ\varphi_\mathrm{com}\circ\hat{\red{\varphi}}_{\red{\mathrm{sk}}}$.
            \item Compute an ideal $J \sim I$ s.t. $\norm(J)$ is $(2^a, 2^b)_3$-nice\\
                by \blue{\textsf{LatticeEnumeration}}.
        \end{itemize}
    }
    \only<5>{
        \textbf{Prover:}
        \begin{itemize}
            \item Let $\varphi_{\mathrm{rsp}}$ be the isogeny corresponding to $J$. (Not computed.)
            \item Let $\red{\alpha} \coloneqq \varphi_\mathrm{sk}\circ\hat{\varphi}_\mathrm{com}\circ\hat{\varphi}_\mathrm{chl}\circ\varphi_\mathrm{rsp} \in \End(E_{\mathrm{pk}})$.
                (by $I$ and $J$)
        \end{itemize}
    }
    \only<6>{
        \textbf{Prover:}
        \begin{itemize}
            \item Compute the images of a basis of $E_\mathrm{pk}[2^a]$ under $\varphi_\mathrm{rsp}$ by
            \begin{align*}
                \varphi_\mathrm{rsp}([2^b]P')
                    = (N_\mathrm{sk}N_\mathrm{com})^{-1}
                        \varphi_\mathrm{chl}\circ\varphi_\mathrm{com}\circ
                        \hat{\varphi}_\mathrm{sk}\circ\red{\alpha}([2^b]P')
            \end{align*}
            for $P' \in E_\mathrm{pk}[2^{a+b}]$.
        \end{itemize}
    }
    \only<7>{
        \textbf{Prover:}
        \begin{itemize}
            \item Compute a $q(2^a - q)$-isogeny $\tau$ by \blue{\textsf{GenRandIsogImages}}.
            \item Compute a $(2^a - q)$-isogeny $\varphi_\mathrm{aux}$ by Kani's lemma.
        \end{itemize}
    }
    \only<8>{
        \textbf{Prover:}
        \begin{itemize}
            \item Let $(P, Q)$ be a fixed basis of $E_\mathrm{aux}[2^a]$.
            \item Send $E_\mathrm{aux}, \varphi_\mathrm{rsp}\circ\hat{\varphi}_\mathrm{aux}(P), \varphi_\mathrm{rsp}\circ\hat{\varphi}_\mathrm{aux}(Q)$ to the verifier.
        \end{itemize}
    }
    \only<9>{
        \textbf{Verifier:}
        \begin{itemize}
            \item Compute $\varphi_\mathrm{rsp}$ by Kani's lemma.
            \item Check if $\varphi_\mathrm{rsp}$ is an isogeny from $E_\mathrm{pk}$ to $E_\mathrm{chl}$.
        \end{itemize}
    }

\end{frame}

\begin{frame}{Failure in finding $\varphi_\mathrm{rsp}$}
    The protocol could \red{fail} in finding $\varphi_\mathrm{rsp}$.\\[3pt]
    \textit{I.e.}, there is no isogeny $E_\mathrm{pk} \to E_\mathrm{chl}$ whose degree is $(2^a, 2^b)_3$-nice.

    \vspace{10pt}
    To reduce the failure probability, we use $q$ s.t.
    $$
        \exists d \mid f \text{ s.t. } q/d \text{ is $(2^a, 2^b)_3$-nice}.
    $$
    (\gray{Recall $p = 2^{a + b} \cdot f - 1$.})
    We say such $q$ is \blue{\emph{$(2^a, 2^b, f)$-nice}}.

    \vspace{10pt}
    The failure probabilities in our parameters is estimated:
    \begin{table}
        \begin{center}
            \begin{tabular}{c|c|c|c|c}
                Security & $a$ & $b$ & $f$ & $log_2(\text{Prob.})$ \\
                \hline
                128 & 129 & 127 & 45 & $\approx -46$ \\
                192 & 191 & 189 & 35 & $\approx -35$ \\
                256 & 254 & 252 & 375 & $\approx -43$
            \end{tabular}
        \end{center}
    \end{table}
    If a failure occurs, the prover retakes $\varphi_{\mathrm{com}}$.
\end{frame}

\newcounter{security_cnt}
\regtotcounter{security_cnt}
\newcommand*{\securitytotal}{\total{security_cnt}}
\stepcounter{security_cnt}
\begin{frame}{Security (\arabic{security_cnt}/\securitytotal{})}
    \begin{prob}[Finding endomorphism]\label{prob:find_endo}
        Given $E$ find $\alpha \in \End(E)\setminus\Z$.
    \end{prob}

    \begin{prob}\label{prob:rii}
        Given $B > 0$, distinguish
        \begin{itemize}
            \item uniform sampling from $\{E \mid \exists\varphi: E_0 \to E \text{ s.t. } \deg\varphi < B\}$,
            \item $E$ computed by \blue{\textsf{RandIsogImages}} with $d < B$. 
        \end{itemize}
    \end{prob}

    \vspace{10pt}
    \begin{itemize}
        \item Problem \ref{prob:find_endo} is standard in isogeny-based cryptography.
        \item A similar problem to Problem \ref{prob:rii} is in QFESTA \cite{C:NakOnu24}.
    \end{itemize}
\end{frame}

\stepcounter{security_cnt}
\begin{frame}{Security (\arabic{security_cnt}/\securitytotal{})}
    To prove the \blue{zero-knowledge property},\\
    we allow the simulator access to the following \blue{oracles}:

    \begin{oracle}[FIxed Degree Isogeny Oracle (FIDIO)]\label{oracle:fidio}
        Given $E$ and an integer $N$, output $\varphi: E \to E'$ of degree $N$.
    \end{oracle}

    \vspace{5pt}
    \begin{oracle}[{\small Random Uniform Nice Degree Isogeny Oracle} (RUNDIO)]\label{oracle:rundio}
        Given $E$, the oracle executes
        \begin{enumerate}
            \item uniformly sample $E'$ from all supersingular elliptic curves,
            \item if $\exists\varphi: E \to E'$ s.t. $\deg\varphi$ is $(2^a, 2^b, f)_3$-nice,\\
                \quad output one of such $\varphi$ uniformly at random,\\
                otherwise,\\
                \quad output the zero morphism.
        \end{enumerate}
    \end{oracle}
\end{frame}

\stepcounter{security_cnt}
\begin{frame}{Security (\arabic{security_cnt}/\securitytotal{})}
    \begin{prob}\label{prob:aux}
        Given $E_\mathrm{pk}$ and a $(2^a, 2^b)_3$-nice integer $q$, distinguish
        \begin{enumerate}
            \item uniform sampling from $\{E \mid \exists\varphi: E_\mathrm{pk} \to E \text{ s.t. } \deg\varphi = 2^a - q\}$,
            \item $E_\mathrm{aux}$ in SQIsign2D-East.
        \end{enumerate}
    \end{prob}

    \vspace{5pt}
    \begin{prob}\label{prob:simu}
        Given $E$, distinguish
        \begin{enumerate}
            \item uniform sampling from all supersingular elliptic curves over $\F_{p^2}$,
            \item $F$ sampled by the following process:
            \begin{enumerate}
                \normalsize
                \item $E' \leftarrow \textsf{RUNDIO}(E)$, rejecting the zero morphism.
                \item $\sigma : E' \to F$ be a uniformly random $2^b$-isogeny.
                \item output $F$.
            \end{enumerate}
        \end{enumerate}
    \end{prob}
\end{frame}

\stepcounter{security_cnt}
\begin{frame}{Security (\arabic{security_cnt}/\securitytotal{})}
    \begin{theorem}[Soundness]
        The identification protocol in SQIsign2D-East is
        \blue{sound}\\
        if Problem \ref{prob:find_endo} and Problem \ref{prob:rii} are hard.
    \end{theorem}

    \vspace{10pt}
    \begin{theorem}[Zero-knowledge]
        The identification protocol in SQIsign2D-East is \blue{zero-knowledge}\\
        if Problem \ref{prob:rii}, Problem \ref{prob:aux}, and Problem \ref{prob:simu} are hard\\
        if the simulator can access to the oracles in Oracle \ref{oracle:fidio} and Oracle \ref{oracle:rundio}.
    \end{theorem}
\end{frame}

\begin{frame}{Signature size}
    
    \textbf{Signature sizes of variants of SQIsign} (in bytes)
    \begin{table}
        \begin{center}
            \begin{tabular}{c|>{\centering\arraybackslash}m{1.5cm}|>{\centering\arraybackslash}m{1.5cm}|>{\centering\arraybackslash}m{1.5cm}|>{\centering\arraybackslash}m{1.5cm}}
                Security & SQIsign & HD & West & \textbf{East} \\
                \hline
                128 & 177 & 109 & 149 & \textbf{154} \\
                192 & 263 & - & 222 & \textbf{223} \\
                256 & 335 & - & 294 & \textbf{296}
            \end{tabular}
        \end{center}
    \end{table}

    \vspace{10pt}
    The public key is $E_\mathrm{pk}$ in all cases.
\end{frame}

\begin{frame}{Performance}

    \textbf{Performance of SQIsign2D-East} (in seconds)
    \begin{table}
        \begin{center}
            \begin{tabular}{c|>{\centering\arraybackslash}m{1.5cm}|>{\centering\arraybackslash}m{1.5cm}|>{\centering\arraybackslash}m{1.5cm}}
                Security & Keygen & Sign & Verify \\
                \hline
                128 & 0.64 & 1.90 & 0.34 \\
                192 & 0.95 & 3.40 & 0.55 \\
                256 & 1.68 & 5.21 & 0.85
            \end{tabular}
        \end{center}
    \end{table}

    \vspace{10pt}
    \begin{itemize}
        \item Proof-of-concept implementation in Julia.
        \item Run on Intel Core i7-10700K CPU @ 3.70GHz.
    \end{itemize}
\end{frame}