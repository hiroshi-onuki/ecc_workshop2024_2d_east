
\begin{frame}{Notation}
    \begin{itemize}
        \setlength{\itemsep}{10pt}
        \item $\lambda$: security parameter.
        \item $p$ : a prime number of the form $2^{a + b} \cdot f - 1 \approx 2^{2\lambda}$,\\[3pt]
                where $b \leq a \leq b + 2$ and $f$ is a small integer.
        \item $E_0 : y^2 = x^3 + x$, a supersingular elliptic curve over $\Fpp$.
        %\item $\B = \Q + \Q \qi + \Q \qj + \Q \qi \qj$ for $\qi^2 = -1$ and $\qj^2 = -p$.
        %\item $\order_0 = \Z + \Z\qi + \Z\frac{\qi + \qj}{2} + \Z\frac{1 + \qi\qj}{2}$: a maximal order of $\B$,\\[3pt]
        %            $\order_0 \cong \End(E_0)$ (we identify $\order_0$ with $\End(E_0)$).
    \end{itemize}

    \vspace{10pt}
    $\End(E_0)$ has "good" properties for isogeny-based cryptography.

    \vspace{20pt}
    All the elliptic curves in this talk are supersingular and over $\Fpp$.
\end{frame}

\newcounter{cnt_quat}
\regtotcounter{cnt_quat}
\newcommand*{\quatotal}{\total{cnt_quat}}
\stepcounter{cnt_quat}
\begin{frame}{Quaternion algebra (\arabic{cnt_quat}/\quatotal{})}
    $\B$ : the \textbf{quaternion algebra} over $\Q$ ramified exactly at $p$ and $\infty$.
    $$\B \cong \Q + \Q\qi + \Q\qj + \Q\qi\qj \quad (\qi^2 = -1,\ \qj^2 = -p,\ \qi\qj = -\qj\qi)$$
    An \textbf{order} of $\B$ is a subring in $\B$ of rank $4$ over $\Z$.\\
    A \textbf{maximal order} is a maximal order of $\B$ w.r.t. inclusion.

    \vspace{10pt}
    The \textbf{canonical involution} $\bar{\phantom{a}}$ on $\B$ is defined by
    $$\overline{a + b\qi + c\qj + d\qi\qj} = a - b\qi - c\qj - d\qi\qj.$$

    For $\alpha = a + b\qi + c\qj + d\qi\qj \in \B$,
    the \textbf{norm} $\norm(\alpha)$ is
    $$\norm(\alpha) \coloneqq \alpha\bar{\alpha} = a^2 + b^2 + p(c^2 + d^2).$$
\end{frame}

\stepcounter{cnt_quat}
\begin{frame}{Quaternion algebra (\arabic{cnt_quat}/\quatotal{})}
    \begin{lemma}
        If $\alpha$ is in an order in $\B$ then $\norm(\alpha) \in \Z_{\geq 0}$.
    \end{lemma}

    \vspace{10pt}
    Let $\order$ be a maximal order in $\B$
    and $I$ be a left $\order$-ideal.

    The \textbf{norm} of $I$ is
    \begin{align*}
        \norm(I) &\coloneqq \gcd\{\norm(\alpha) \mid \alpha \in I\}.
    \end{align*}

    The \textbf{conjugate} of $I$ is
    \begin{align*}
        \bar{I} &\coloneqq \{\bar{\alpha} \mid \alpha \in I\}.
    \end{align*}

    Let $\order'$ be another maximal order in $\B$.\\
    If $I$ is a right $\order'$-ideal then $I$ is a \textbf{connecting ideal} of $\order$ and $\order'$.
\end{frame}

\stepcounter{cnt_quat}
\begin{frame}{Quaternion algebra (\arabic{cnt_quat}/\quatotal{})}
    \begin{lemma}
        Let $I$ be a connecting ideal of $\order$ and $\order'$.\\
        Then, $\bar{I}$ is a connecting ideal of $\order'$ and $\order$.
    \end{lemma}

    \vspace{10pt}
    \begin{definition}
        Let $I$ and $J$ be left $\order$-ideals.\\
        $I$ and $J$ are \textbf{equivalent}
        if there exists $\beta \in \B^{\times}$ such that $I = J\beta$.\\
        We denote $I \sim J$.
    \end{definition}
\end{frame}

\begin{frame}{Deuring correspondence}

    {\fontsize{11pt}{13pt}\selectfont
    $\mathcal{E}_p \coloneqq \{\overline{\F}_p\mbox{-isomorphisim classes of supersingular elliptic curves over }\Fpp \}$\\
    \begin{itembox}{\blue{\textbf{Deuring Correspondence}}}
        \begin{center}
            \begin{tabular}{ccc}
                $\mathcal{E}_p/\mathrm{Gal}(\Fpp/\Fp)$ & $\stackrel{1:1}{\longleftrightarrow}$
                    & \{Maximal orders $\subset \B$\}$/\cong$\\[5pt]\hline\\[5pt]
                $E$  & $\longleftrightarrow$
                    & $\order$ s.t. $\order \cong \End(E)$\\[7pt]
                An isogeny $\varphi: E_1 \to E_2$ & $\longleftrightarrow$
                    & $I$ : a connecting ideal of $\order_1$ and $\order_2$\\[7pt]
                $\deg\varphi_I$ & $\longleftrightarrow$ & $\norm(I)$\\[7pt]
                $\xymatrix{
                    E_0 \ar@/^15pt/[r]^{{\varphi_I}} \ar@/_15pt/[r]_{{\varphi_J}} & E_1
                }$ & $\longleftrightarrow$ & $I \sim J$
            \end{tabular}
        \end{center}
    \end{itembox}
    }
\end{frame}
\begin{frame}{SQIsign}
    \begin{itemize}
        \setlength{\itemsep}{10pt}
        \item An isogeny-based signature scheme by \cite{AC:DKLPW20}.
        \item A candidate for the NIST PQC additional signatures.
        \item Use the \blue{Deuring correspondence}.
        \item Use Fiat-Shamir transform to the \blue{identification protocol}\\ in the next slide.
    \end{itemize}
\end{frame}

\begin{frame}{SQIsign}
    \begin{center}
        Proof of knowledge of the secret isogeny $\red{\varphi_\mathrm{sk}}$.
    \end{center}

    \vspace{-10pt}
    
    {\large
    \begin{equation*}
        \xymatrix@C=50pt@R=50pt{
            E_0 \ar@{..>}[d]_{\red{\varphi_\mathrm{sk}}} \only<2->{\ar@{..>}[r]^{\varphi_{\mathrm{com}}}}
                & \uncover<2->{E_{\mathrm{com}}} \only<3->{\ar[d]^{\varphi_\mathrm{chl}}}\\
            E_{\mathrm{pk}} \only<5->{\ar[r]_{\varphi_\mathrm{rsp}}} & \uncover<3->{E_{\mathrm{chl}}}
        }
    \end{equation*}
    }

    \vspace{10pt}
    \only<2>{
        \textbf{Prover:}
        \begin{itemize}
            \item Compute a random isogeny $\varphi_{\mathrm{com}}$ and the codomain $E_{\mathrm{com}}$.
            \item Send $E_{\mathrm{com}}$ to the verifier.
        \end{itemize}
    }
    \only<3>{
        \textbf{Verifier:}
        \begin{itemize}
            \item Compute a random isogeny $\varphi_{\mathrm{chl}}$ and the codomain $E_{\mathrm{chl}}$.
            \item Send $\varphi_\mathrm{chl}$ and $E_{\mathrm{chl}}$ to the prover.
        \end{itemize}
    }
    \only<4>{
        \textbf{Prover:}
        \begin{itemize}
            \item Compute the ideal $I$ corresponding to $\varphi_\mathrm{chl}\circ\varphi_\mathrm{com}\circ\hat{\red{\varphi}}_{\red{\mathrm{sk}}}$.
            \item Compute an ideal $J$ equivalent to $I$ whose norm is a power of $2$.
        \end{itemize}
    }
    \only<5>{
        \textbf{Prover:}
        \begin{itemize}
            \item Compute the isogeny $\varphi_{\mathrm{rsp}}$ corresponding to $J$.
            \item Send $\varphi_{\mathrm{rsp}}$ to the verifier.
        \end{itemize}
    }
    \only<6>{
        \textbf{Verifier:}
        \begin{itemize}
            \item Check if $\varphi_{\mathrm{rsp}}$ is an isogeny from $E_{\mathrm{pk}}$ to $E_{\mathrm{chl}}$\\
                \hphantom{Check if} and the kernel of $\hat{\varphi}_{\mathrm{chl}}\circ\varphi_{\mathrm{rsp}}$ is cyclic.
        \end{itemize}
    }
\end{frame}

\begin{frame}{Kani's lemma}
    \begin{lemma}[\cite{Kan1997number,EC:MMPPW23}]
        Consider a commutative diagram of isogenies
        \vspace{-10pt}
        \begin{align*}
            \xymatrix@C=30pt@R=30pt{
                E \ar[r]^{\varphi_1} \ar[d]_{\varphi_2} & E_1 \ar[d]^{\psi_2}\\
                E_2 \ar[r]_{\psi_1} & F
            }
            \qquad
            \xymatrix@R=0.3em{
                \deg\varphi_1 = \deg\psi_1 = d_1,\\
                \deg\varphi_2 = \deg\psi_2 = d_2,\\
                d = d_1 + d_2.
            }
        \end{align*}

        Then, the isogeny with kernel
        $$\{\varphi_1(R), \varphi_2(R) \mid R \in E[d]\}$$
        is a $(d,d)$-isogeny$E_1 \times E_2 \to E \times F$ and represented by
        \vspace{-10pt}
        \begin{equation*}
            \begin{pmatrix}
                \hat{\varphi_1} & \hat{\varphi_2}\\
                -\psi_2 & \psi_1
            \end{pmatrix}.
        \end{equation*}
    \end{lemma}
\end{frame}

\begin{frame}{RandIsogImages (RII)}
    \begin{algorithm}[H]
        \caption{\textsf{RandIsogImages} \cite{C:NakOnu24}}
        \KwIn{An odd integer $d < 2^{a+b}$.}
        \KwOut{A $d$-isogeny from $E_0$.}
        \BlankLine
        Let $\alpha \leftarrow \textsf{FullRepresentInteger}_{\order_0}(d\cdot(2^{a+b} - d))$\;
        Compute a $d$-isogeny $\varphi$ by Kani's lemma\;
        \Return{$\varphi$}\;
    \end{algorithm}

    \vspace*{5pt}
    \begin{equation*}
        \xymatrix{
            E_0 \ar[r]^\varphi \ar[d]_\psi \ar[dr]^{\alpha} & E_1 \ar[d]^{\psi'} \\
            E_2 \ar[r]_{\varphi'} & E_0,
        }
        \hspace*{10pt}
        \xymatrix@R=5pt{
            \\
            \deg\varphi = \deg\varphi' = d,\\
            \deg\psi = \deg\psi' = 2^{a+b} - d.
        }
    \end{equation*}
\end{frame}