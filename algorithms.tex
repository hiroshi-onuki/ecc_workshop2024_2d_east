\begin{frame}{Requirement on $\varphi_\mathrm{aux}$}
    \begin{itemize}
        \setlength{\itemsep}{10pt}
        \item $\deg\varphi_\mathrm{res} + \deg\varphi_\mathrm{aux} = 2^a \approx p^{1/2}$\\[3pt]
            $\Rightarrow \deg\varphi_\mathrm{aux} < 2^a$.\\[3pt]
            \hphantom{$\Rightarrow$} $\deg\varphi_\mathrm{aux}$ is nonsmooth.
        \item The domain $E_\mathrm{pk}$ is not a special curve\\[3pt]
            $\Rightarrow$ we cannot use \blue{\textsf{FullRepresentInteger}}.
        \item The prover knows
                \begin{align*}
                    \xymatrix@C=30pt{
                        E_0 \ar[r]^{\varphi_\mathrm{sk}} & E_\mathrm{pk}
                    }.
                \end{align*}
            $\Rightarrow$ we can use \blue{\textsf{SpecialEichlerNorm}}.
    \end{itemize}

\end{frame}

\begin{frame}{generalized RandIsogImages}
    Replaicing \blue{\textsf{FullRepresentInteger}} with \blue{\textsf{SpecialEichlerNorm}},\\
    we obtain a new algorithm \blue{\textsf{GenRandIsogImages}}.

    \vspace{10pt}
    \begin{algorithm}[H]
        \caption{\textsf{GenRandIsogImages}}
        \KwIn{
            An elliptic curve $E$
            and an odd integer $d < 2^{a+b}$,\\
            \hphantom{Input: }
            \red{an isogeny $\varphi_I: E_0 \to E$ of prime degree $N$} s.t.\\
            \hphantom{Input: }
            \red{$d(2^{a+b} - d) > p\cdot N^3$}
            and \red{$\left(\frac{-d(2^{a+b} - d)}{N}\right) = 1$}.
        }
        \KwOut{A $d$-isogeny $\varphi$ from $E$.}
        \BlankLine
        Let $\alpha \leftarrow \red{\textsf{SpecialEichlerNorm}_{I}(E, d(2^{a+b} - d))}$\;
        Compute a $d$-isogeny $\varphi$ by Kani's lemma\;
        \Return{$\varphi$}\;
    \end{algorithm}
\end{frame}

\newcounter{auxRIIcnt}
\regtotcounter{auxRIIcnt}
\newcommand*{\auxRIItotal}{\total{auxRIIcnt}}
\stepcounter{auxRIIcnt}
\begin{frame}{Auxiliary isogeny (\arabic{auxRIIcnt}/\auxRIItotal{})}
    Consider applying \blue{\textsf{GenRandIsogImages}} to $E_\mathrm{pk}$.
    \begin{align*}
        d(2^{a+b} - d) > p (\deg\varphi_\mathrm{sk})^3.
    \end{align*}
    \mypause
    Since $2^{a+b} - d \approx p$, we have
    \begin{align*}
        d > (\deg\varphi_\mathrm{sk})^3.
    \end{align*}

    \mypause
    If we sample $E_\mathrm{pk}$ from all supersingular curves
    then $\deg\varphi_\mathrm{sk} \approx p^{1/2}$.

    \mypause
    \vspace{10pt}
    Restricting $E_\mathrm{pk}$ to a \textbf{subset used in SQIsign}
    then $\deg\varphi_\mathrm{sk} \approx p^{1/4}$.

    \vspace{5pt}
    We use this restriction, but
    $$ d > p^{3/4}$$
    is \red{too large} for $\varphi_\mathrm{aux}$ (we want $\deg\varphi_\mathrm{aux} < 2^a$).

\end{frame}

\stepcounter{auxRIIcnt}
\begin{frame}{Auxiliary isogeny (\arabic{auxRIIcnt}/\auxRIItotal{})}
    Let $q \coloneqq \deg\varphi_\mathrm{rsp}$.
    (We take $q$ as a random odd integer s.t. $q < 2^a$.)
    $$
        \xymatrix@C=40pt@R=40pt{
            E_\mathrm{aux} \ar[r]^\psi & E_1\\
            E_\mathrm{pk} \ar[u]^{\magenta{\varphi_\mathrm{aux}}} \ar[ur]^{\red{\tau}} \ar[r] & E_2 \ar[u]
        }
    $$
    \begin{enumerate}
        \setlength{\itemsep}{10pt}
        \item Compute a $d$-isogeny \red{$\tau$} by \blue{\textsf{GenRandIsogImages}}
            with $d = q(2^a - q) \approx p$.
        \item Divide $\red{\tau} = \psi \circ \magenta{\varphi_\mathrm{aux}}$
            s.t. $\deg\psi = q$ and $\deg\magenta{\varphi_\mathrm{aux}} = 2^a - q$\\
            by Kani's lemma.
    \end{enumerate}
\end{frame}

\stepcounter{auxRIIcnt}
\begin{frame}{Auxiliary isogeny (\arabic{auxRIIcnt}/\auxRIItotal{})}

    \vspace{-10pt}
    \begin{align*}
        M(q) &\coloneqq q(2^a - q)(2^{a + b} - q(2^a - q))\\
            &= \deg\alpha \text{ in \blue{\textsf{GenRandIsogImages}} with } d = q(2^a - q).
    \end{align*}

    \mypause
    \vspace{10pt}
    Requirements for $q = \deg\varphi_\mathrm{rsp}$ and $N = \deg\varphi_\mathrm{sk}$:
    \begin{enumerate}
        \item $q$ is odd
        \item $q < 2^a$
        \item $q(2^a - q) < 2^{a + b}$\quad (follows from \textbf{2} and $a \leq b + 2$)
        \item $p \cdot N^3 < M(q)$\qquad (follows from $N < p^{1/4}$)
        \item $\left(\frac{-M(q)}{N}\right) = 1$
    \end{enumerate}

    \vspace{10pt}
    \red{Issue} : $q$ depends on $N$.\quad
    (pointed out by \cite{CCILV2024sqisign2d_attack})
\end{frame}

\begin{frame}{Restriction on $E_\mathrm{pk}$}
    The space of the secret keys
    $$
        \mathcal{S}_\mathrm{sk} = \{
            \varphi_\mathrm{sk}: E_0 \to \cdot
            \mid \deg\varphi_{\mathrm{sk}} < p^{1/4} \}.
    $$
    \begin{itemize}
        \setlength{\itemsep}{10pt}
        \item The number of possible choices of $N \coloneqq \deg\varphi_\mathrm{sk}$ is about $p^{1/4} \approx 2^{\lambda/2}$.
        \item $N \approx p^{1/4} \approx 2^{\lambda/2} \Rightarrow$ there are about $2^{\lambda/2}$ $N$-isogenies from $E_0$.
        \item $\#\mathcal{S}_\mathrm{sk} \approx 2^{\lambda}$.
        \item \red{The degree $N$ must be kept secret.}\\
            Otherwise, the brute-force attack with $N$-isogenies costs $2^{\lambda/2}$.
    \end{itemize}

\end{frame}

\newcounter{countermeasurecnt}
\regtotcounter{countermeasurecnt}
\newcommand*{\countermeasuretotal}{\total{countermeasurecnt}}
\stepcounter{countermeasurecnt}
\begin{frame}{Modification (\arabic{countermeasurecnt}/\countermeasuretotal{})}
    \textbf{Idea:}
    making $q$ independent of $\deg\varphi_\mathrm{sk}$.

    \vspace{10pt}
    \textbf{Solution:}
    \begin{itemize}
        \item $\deg\varphi_\mathrm{sk}$ is sampled from
            a prime $N < p^{1/4}$ s.t. \red{$\left(\frac{3}{N}\right) = -1$}.
        \item Change the condition $\left(\frac{-M(q)}{N}\right) = 1$ to
            \red{$3 \mid M(q)$}.
        \item Modify \blue{\textsf{GenRandIsogImages}} as in the next slide.
    \end{itemize}

    \mypause
    \vspace{10pt}
    We say $q$ is \blue{\emph{$(2^a, 2^b)_3$-nice}} if
    \begin{itemize}
        \item $q$ is odd,
        \item $q < 2^a$,
        \item $q(2^a - q) < 2^{a + b}$,
        \item $3 \mid M(q)$.
    \end{itemize}

    \vspace{5pt}
    We make $\deg\varphi_\mathrm{rsp}$ $(2^a, 2^b)_3$-nice.
\end{frame}

\stepcounter{countermeasurecnt}
\begin{frame}{Modification (\arabic{countermeasurecnt}/\countermeasuretotal{})}
    \begin{algorithm}[H]
        \caption{(modified) \textsf{GenRandIsogImages}}
        \KwIn{
            An elliptic curve $E$
            and an odd integer $q < 2^a$,\\
            \hphantom{Input: }
            an isogeny $\varphi_I: E_0 \to E$ of prime degree $N$ s.t.\\
            \hphantom{Input: }
            $M(q) > p\cdot N^3$,
            \red{$\left(\frac{3}{N}\right) = -1$},
            and \red{$3 \mid M(q)$}.
        }
        \KwOut{A $q(2^a - q)$-isogeny $\varphi$ from $E$.}
        \BlankLine
        \If{$\left(\frac{-M(q)}{N}\right) = 1$}{
            Let $\alpha \leftarrow \textsf{SpecialEichlerNorm}_{I}(E, M(q))$\;
        }
        \Else{
            Let $\alpha' \leftarrow \textsf{SpecialEichlerNorm}_{I}(E, M(q)\red{/3})$\;
            \red{Compute a random $3$-isogeny $\varphi_3: E \to E_3$}\;
            \red{Let $\alpha \leftarrow \varphi_3 \circ \alpha'$}\;
        }
        Compute a $q(2^a - q)$-isogeny $\varphi$ by Kani's lemma\;
        \Return{$\varphi$}\;
    \end{algorithm}
\end{frame}
