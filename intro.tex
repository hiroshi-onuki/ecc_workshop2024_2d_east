\newcounter{intro}
\regtotcounter{intro}
\newcommand*{\introTotal}{\total{intro}}
\stepcounter{intro}
\begin{frame}{Introduction (\arabic{intro}/\introTotal{})}
    The talk is about 

    \begin{itemize}
        \setlength{\itemsep}{10pt}
        \item K. Nakagawa and \underline{O},\\
            "SQIsign2D-East: A New Signature Scheme Using 2-dimensional Isogenies",\\
            ePrint 2024/771
            \cite{no2024sqisign2d}.

        \item W. Castryck, M. Chen, R. Invernizzi, G. Lorenzon, F. Vercauteren,\\
                "Breaking and Repairing SQIsign2D-East",\\
                ePrint 2024/1453
                \cite{CCILV2024sqisign2d_attack}
    \end{itemize}

    \vspace{10pt}
    The \textbf{merged paper} will be appeared in Asiacrypt 2024.
\end{frame}

\stepcounter{intro}
\begin{frame}{Introduction (\arabic{intro}/\introTotal{})}
    \begin{itemize}
        \setlength{\itemsep}{10pt}
        \item \blue{SQIsign2D-East} is a new signature scheme using 2-dimensional isogenies based on SQIsign.
        \item "\textbf{East}" is for distinguishing from
                \textbf{SQIsign2D-West} proposed by
                A. Basso, L. De Feo, P. Dartois, A. Leroux, L. Maino,
                G.  Pope, D. Robert, B. Wesolowski
                \cite{BDDLMPRW2024sqisign2d}.
        \item The original \blue{SQIsign2D-East} was proposed by \cite{no2024sqisign2d}.
        \item A \red{security issue} was found by \cite{CCILV2024sqisign2d_attack}.
        \item A method to \blue{repair} it was also proposed by \cite{CCILV2024sqisign2d_attack}.
    \end{itemize}
\end{frame}

